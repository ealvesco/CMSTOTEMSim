\input /home/jkaspar/tex/kaspiTeX/base
\input /home/jkaspar/tex/kaspiTeX/article
\input /home/jkaspar/tex/kaspiTeX/biblio

\Reftrue

%\parindent=0pt
%\parskip=3pt plus5pt


%%%%%%%%%%%%%%%%%%%%%%%%%%%%%%%%%%%%%%%%%%%%

\BeginText

\title{Energy, angular and vertex smearing}

\section{Goal}
Monte Carlo simulators usually consider head-to-head collision of protons with nominal energy and momentum parallel to the $z$ axis. This is not precisely the actual case due to the energy and angular smearing and also due to the crossing angle. The aim of this note is to show how to correct momenta of particles generated by a MC to be complaint with the real-life situation. Secondly, most MC generators place the collision to the origin while due to the bunch sizes the actual vertices are distributed around the system origin. And therefore, I tried to calculate vertex distribution in the case of non-zero crossing angle.

\section{Conventions}

Proton 1 goes from lower $z$ to higher $z$. Angles are measured from beam axis in CCW direction, i.e. $\al_1$ is negative and $\al_2$ is positive in the Figure 1.

\fig[10cm]{fig/fig1d.eps}{crossing angle}{Sketch of energy/angular smearing.}

\section{Energy-angular smearing}
There are three contributions: energy smearing, the beam-divergence and the crossing angle. Smearing in energy means that a proton has momentum $p = (1 + \xi)\, p_0$ where $p_0$ is the nominal momentum ($7\un{TeV}$).

\fig[8cm]{fig/fig2d.eps}{beam divergence}{Beam divergence illustration.}

The beam divergence phenomenon is sketched in Figure 2. For simplicity the original (not smeared) momentum is considered parallel to the $z$ axis. Due to the beam divergence, actual momenta would form a cone around the original momentum. The distribution in $\th$ is Gaussian while $\ph$ has uniform distribution. For given values of $\th$ and $\ph$ the smeared proton has momentum

\eqref{\vec p = |\vec p|\,(\cos\th\cos\ph, \sin\th\sin\ph, \cos\th) \.}{bd}

The crossing-angle takes place in $x-z$ plane as shown in Figure 1. Momentum of the proton 1 is
\eqref{\vec p_1 = \pmatrix{\cos\al_1 & 0 & \sin\al_1\cr 0 & 1 & 0 \cr -\sin\al_1 & 0 & \cos\al_1\cr} \pmatrix{0\cr 0\cr p_1\cr} \.}{ca}

The combination of all these effects is straightforward, e.g. for proton 1 it gives
\eqref{\vec p_1 = p_0\,(1 + \xi_1)\,\pmatrix{\cos\al_1 & 0 & \sin\al_1\cr 0 & 1 & 0 \cr -\sin\al_1 & 0 & \cos\al_1\cr}\,\pmatrix{\sin\th\cos\ph\cr \sin\th\sin\ph\cr \cos\th}\.}{all}
For the second proton we get an analogous expression, the only difference is overall minus sign following from the opposite direction.

The ceter-of-mass (CM) frame moves in laboratory (LAB) frame in direction $\vec d$ and velocity $\be$,
\eqref{\vec d = \vec p_1 + \vec p_2,\qquad \beta = {|\vec p_1 + \vec p_2|\over E_1 + E_2} ,}{beta}
where $E_1$ and $E_2$ are (LAB) energies of the protons. To transform a four vector $(E| \vec p)$ from LAB to CM, one applies Lorentz transformation
\eqref{(E'|\vec p') = L(\vec d, -\beta) (E|\vec p)\.}{lorentz}
After this transformation, vectors $\vec p_1'$ and $\vec p_2'$ have the same magnitude and the opposite direction. However, they need not be collinear with $z$ axis. Hence to connect output of MC generators with the real-life situation one has apply a rotation. The rotation has axis $\vec a$ and angle $\om$
\eqref{\vec a = \vec p_1' \times (0, 0, 1), \qquad \cos\om = {\vec p_1' \cdot (0, 0, 1)\over |\vec p_1'|}}{rotation}

Putting all together, the transformation between the LAB and the MC systems reads
\eqref{(E|\vec p)|_{MC} = R(\vec a, \om)\,L(\vec d, -\beta)\, (E|\vec p)|_{LAB}\.}{lab to mc}
To transform products of MC to the LAB frame, one can use the inverse transformation
\eqref{(E|\vec p)|_{LAB} = L(\vec d, \beta)\,R(\vec a, -\om)\, (E|\vec p)|_{MC}\.}{mc to lab}
The $4\times4$ matrix $L(\vec d, \beta)\,R(\vec a, -\om)$ can be expressed as a function of $\al, \th, \ph, \xi$ for protons 1 and 2 but the dependency is rather complicated to put here.

Unfortunately, this is not the full story. Again, due to the energy and angular smearing, the CM energy $\sqrt{s}$ differs from the nominal $\sqrt{s_0} = 14\un{TeV}$. Generally, there is no way to connect events at different energies. These events are simply different events. However, the difference is CM energy is tiny. It is of order of $\xi$ or $\th$, hence $10^{-4}$. On this scale one may neglect $s$-dependency of the angular distribution, particle distribution, etc. In other words, energies of MC products can be simply scaled to sum up to the correct $\sqrt{s}$ value:
\eqref{E_i \rightarrow E_i' = E_i\, (1 + \bar\xi), }{energy scaling}
where $\bar\xi$ is defined $\sqrt{s} = \sqrt{s_0}\,(1+\bar\xi)$. If we denote this operation $S(\bar\xi)$, then transformation of particle momentum from MC to real LAB frame can be written
\eqref{(E|\vec p)|_{LAB} = L(\vec d, \beta)\,R(\vec a, -\om)\, S(\bar\xi)\, (E|\vec p)|_{MC}\.}{mc to real}

\section{Vertex smearing}

In this case we consider only the crossing-angle effects. The situation is depicted in Figure 3. Let's suppose that particles within a bunch have a Gaussian distribution, i.e. particle density $\rh$ is
\eqref{\rh(\tilde x, \tilde y, \tilde z) = {n_B\over (2\pi)^{3/2}\si_x\si_y\si_z}\,\exp\left(-{\tilde x^2\over 2\si_x^2}-{\tilde y^2\over 2\si_y^2}-{\tilde z^2\over 2\si_z^2}\right),}{density}
where $n_B$ is number of protons in a bunch.

\fig[10cm]{fig/figure3.eps}{vertex smearing}{Beam collision with crossing angle.}

One of definitions of crossing-section $\si$ relates it to the corresponding number of events $N$ per time
\eqref{{\d N\over \d t} = \si\,j\,n_T\.}{cross-section}
$j$ denotes flux of bombarding particles and $n_T$ is number of target particles. This formula can be written in a more general form
\eqref{N = \si\,L_{int},\qquad L_{int} = \int\d t\,\d x\,\d y\,\d z\, j(x, y, z; t)\, \rh_T(x, y, z; t), }{luminosity}
where $L_{int}$ is integrated luminosity in a given time interval and $\rh_T$ is density of target particles. It is clear that function
\eqref{h(x, y, z; t) = {1\over L_{int}}\,j(x, y, z; t)\, \rh_T(x, y, z; t)}{pdf}
can be regarded as the probability density function of finding vertex at position $(x, y, z)$ and in time $t$.

Now, we enumerate function $h$ for our situation. Beside the assumption \ref{density} we assume that centers of both bunches reach the origin of LAB frame at time $t = 0$. The flux $j$ can be calculated as $v_{rel}\,\rh$, where $v_{rel}$ is relative velocity of the two bunches and $\rh$ is density of one of them. Both bunches move with velocity $v$ (see \Fg{vertex smearing}) and therefore $v_{rel} = 2v\cos\al$. We are left with formula
\eqref{h(x, y, z; t) = {2v\cos\al\over L_{int}}\,\rh_1(x, y, z; t)\, \rh_2(x, y, z; t)\.}{pdf2}
The $\rh$ densities correspond to the two bunches (the formula is symmetric against swap of the two bunches).

After straightforward transformation of coordinates from tilted to non-tilted frame (see \Fg{vertex smearing}), one gains
\eqref{\rh_1(x, y, z; t) = \rh(x\cos\al + z\sin\al, y, -x\sin\al + z\cos\al - vt)}{density1}
\eqref{\rh_2(x, y, z; t) = \rh(x\cos\al - z\sin\al, y, x\sin\al + z\cos\al + vt)}{density2}
the $\rh$ is given by \Eq{density}. Performing some algebra one can find
\eqref{h(x, y, z; t) \propto \exp\left[- {\cos^2\al\over\si^2_x} x^2 - {y^2\over\si^2_y} - \left({\sin^2\al\over\si^2_x} + {\cos^2\al\over\si^2_z}\right)z^2 - {(vt + x\sin\al)^2\over\si^2_z}\right]\.}{vertex full distribution}
In other words, the random variables $x$ and $t$ are not independent. But as we are not interested in the time of collision, we can integrated over time $t$ and obtain p.d.f. only for spatial coordinates of the vertex
\eqref{h(x, y, z) \propto \exp\left[- {\cos^2\al\over\si^2_x} x^2 - {y^2\over\si^2_y} - \left({\sin^2\al\over\si^2_x} + {\cos^2\al\over\si^2_z}\right)z^2\right]\.}{vertex distribution}
We can see the distribution is Gaussian again. The mean values stay zero while the effective variations are
\eqref{\si_{x,ef\!f} = {\si_x\over\sqrt{2}\cos\al},\quad \si_{y,ef\!f} = {\si_y\over\sqrt{2}},\quad \si_{z,ef\!f} = {\si_z\over\sqrt{2}}\,{\si_x\over\sqrt{\si_z^2\sin^2\al + \si_x^2\cos^2\al}} \.}{effective variations}




\EndText
\end
